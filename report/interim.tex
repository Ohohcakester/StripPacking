\documentclass{article}
\usepackage{amsmath}
\usepackage[a4paper]{geometry}

\begin{document}

\author{
Oh Shunhao\\
  \texttt{A0065475X}
  \and
Nguyen Quoc Dat\\
  \texttt{A0116703N}
}
\title{CS4234 - Strip Packing}
\date{}

\maketitle

\begin{abstract}
We demonstrate a polynomial time algorithm for 100-D Bin-packing
\end{abstract}

\section{The Strip Packing Problem}
In the Strip Packing problem, we are given an infinitely tall strip, of maximum width $W$, and a set of $n$ rectangles $s_1,s_2,\cdots,s_n$, each represented by a tuple $s_i = (w_i,h_i)$, representing the width and height of each rectangle.\\
\\
The task is to pack all of the $n$ rectangles into the strip, so that none of the rectangles overlap, and such that the total height of the strip is minimised.\\
\\
The decision problem of Strip Packing easily shown to be NP-HARD via a reduction from the partition problem.\\

Assuming $P \neq NP$, there is no absolute polynomial time approximation scheme for Strip Packing. In fact, there is no algorithm with an absolute approximation ratio better than $\frac{3}{2} - \epsilon$ as 1-dimensional bin-packing is a subproblem of strip packing (1-dimensional bin packing is equivalent to strip packing with rectangles of height $1$), and in 1-dimensional bin packing, it is NP-HARD to distinguish whether OPT is $2$ or $3$ due to a reduction from the Partition problem.




\end{document}